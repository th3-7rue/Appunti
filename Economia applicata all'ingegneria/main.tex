\documentclass{report}

\usepackage[italian]{babel} % Imposta la lingua italiana
\usepackage{tcolorbox}
\usepackage{forest}
\usepackage{dirtree} % Aggiungi il pacchetto dirtree
\usepackage{graphicx} % Aggiungi il pacchetto graphicx
\usepackage{hyperref}
\usepackage{eurosym}
\usepackage{circuitikz}
\usepackage{amsmath}
\usepackage{emoji}
\usepackage{numprint}
\hypersetup{
    colorlinks=true,
    linkcolor=black,
    filecolor=magenta,      
    urlcolor=cyan,
    pdftitle={Overleaf Example},
    pdfpagemode=FullScreen,
    }
% Definizione del nuovo ambiente 'important'
\newtcolorbox{important}[1][]{
  colback=yellow!10!white,
  colframe=red!50!black,
  title=Ricorda,
  #1
}
% Definizione del nuovo ambiente 'formula'
\newtcolorbox{formula}[1][]{
  colback=blue!5!white,
  colframe=blue!75!black,
  title=Formula,
  #1
}
% Definizione del nuovo ambiente 'definition'
\newtcolorbox{definition}[1][]{
  colback=green!5!white,
  colframe=green!75!black,
  title=Definizione,
  #1
}
\begin{document}

\begin{center}
  \vspace*{2cm}
  {\Huge Economia applicata all'ingegneria \par}
  \vspace{1cm}
  \includegraphics[width=0.5\textwidth]{logounibs.png}\par
  \vspace{1cm}
  {\Large Riccardo Rasori \par}
  \vspace{0.5cm}
  {\large A.A. \numprint{2024}/\numprint{2025} \par}
  \vspace{2cm}
\end{center}

\tableofcontents % Aggiungi l'indice

\section{Analisi degli investimenti 1}
\subsection{Introduzione alla matematica finanziara}
\textnormal{Valori nominali $\rightarrow$ anno corrente}
\newline
\textnormal{Valori reali $\rightarrow$ determinato anno (regolato a indice)}
\begin{formula}
  $Vk_t=(VC_t/IPC_t)*100$
\end{formula}
\vspace{1cm}
\dirtree{%
  .1 Tasso di interesse $\rightarrow$ prezzo della moneta $\rightarrow$ controllato da Banca Centrale .
  .2 Nominale .
  .2 Reale $\rightarrow$ $i_k=(ic-\gamma)/(1+\gamma)$  \newline{Nota: $\gamma$ rappresenta il tasso di inflazione.}.
}
\dirtree{%
  .1 La banca centrale controlla il prezzo della moneta .
  .2 Immettendo denaro tramite l'acquisto di titoli di stato $\rightarrow$ riduce il tasso .
  .2 Distruggendo denaro tramite la vendita di titoli di stato $\rightarrow$ aumenta il tasso .
}
\begin{formula}
  $C \rightarrow C(1+r)$\newline
  r $\rightarrow$ tasso di crescita \newline
  C $\rightarrow$ capitale
\end{formula}

\subsection{Costo opportunità del capitale}
\subsection{Interesse e montante semplice}
\begin{formula}
  $I=C*r*t$
\end{formula}
\begin{formula}
  $M=C(1+rt)$
  \newline
  M $\rightarrow$ montante $\rightarrow$ somma del capitale e degli interessi maturati nel tempo t
\end{formula}
Dimostrazione:\newline
$M=C+I=C+Crt=C(1+rt)$
\subsection{Montante semplice di rate stabili}
\begin{formula}
  $M=R(n+\frac{rn \pm 1}{2})$\newline
  +1 se la rata è anticipata, -1 se la rata è posticipata
\end{formula}
Es. 300€ canone mensile, 1,8\% saggio, 12 mesi\\
$M=300(12+0,018*\frac{12+1}{2})=\numprint{3636,10}$\texteuro
\subsection{Interesse e montante composto}
\begin{formula}
  $M_n=C(1+r)^n$  n = numero degli anni\\
  $C=\frac{R}{1+r}$
\end{formula}
$M_1=C(1+r)$\\
$M_2=M_1(1+r)$=$C(1+r)^2$\\
\vspace{0.1cm}

Es. Ho \numprint{1000}\texteuro e li investo con +10\% ogni anno\\
\begin{figure}[!ht]
  \centering
  \resizebox{0.4\textwidth}{!}{%
    \begin{circuitikz}
      \tikzstyle{every node}=[font=\large]
      \draw [ color={rgb,255:red,40; green,32; blue,32}, short] (7.75,9.5) -- (12.75,9.5);
      \draw [short] (7.75,9.25) -- (7.75,9.75);
      \draw [short] (12.75,9.25) -- (12.75,9.75);
      \draw [short] (10.25,9.25) -- (10.25,9.75);
      \node [font=\large] at (7.75,9) {0};
      \node [font=\large] at (10.25,9) {1};
      \node [font=\large] at (12.75,9) {2};
      \node [font=\large] at (13,10) {\numprint{1100}+10\%(\numprint{1100})};
      \node [font=\large] at (10.25,10) {\numprint{1100}};
      \node [font=\large] at (7.75,10) {\numprint{1000}};
    \end{circuitikz}
  }%
  \vspace{0.1cm}\\
  $C=\frac{M_n}{(1+r)^n}$\\
  \vspace{0.1cm}
  $q=1+r$\\
\end{figure}
\subsection{Valore futuro (VF)}
\begin{definition}
  È l'ammontare di una somma di denaro complessiva degli interessi in un determinato periodo.
\end{definition}
\subsection{Composizione degli interessi}
\numprint{1000}\texteuro, 2 anni, 10\%\\
$M_2=\numprint{1000}(1+0,1)^2=\numprint{1210}$\texteuro\\
\textbf{Interesse composto}
\subsection{Valore attuale (VA)}
Quanto devo investire oggi per avere \numprint{2000}\texteuro \space tra un anno con saggio 11\%?\\
$VA=\frac{\numprint{2000}}{1+0,11}=\numprint{1801,80}$\texteuro
\begin{formula}
  $VA=\frac{FV}{(1+r)^n}$
\end{formula}
Desidero avere a disposizione \numprint{10000}\texteuro\space per un viaggio negli States tra 4 anni dopo che mi laureo tutto pelato \emoji{neutral-face}\\
Quale somma dovrò accantonare ogni mese al saggio del 3\%?\\
Quale somma dovrò depositare sul conto corrente bancario oggi?\\
$\frac{\numprint{10000}\texteuro}{(1,03)^4}=\numprint{8884,87}$\texteuro
\subsection{Flussi di cassa multipli}
\begin{figure}[!ht]
  \centering
  \resizebox{0.3\textwidth}{!}{%
    \begin{circuitikz}
      \tikzstyle{every node}=[font=\large]
      \draw [ color={rgb,255:red,40; green,32; blue,32}, short] (7.75,9.5) -- (12.75,9.5);
      \draw [short] (7.75,9.25) -- (7.75,9.75);
      \draw [short] (12.75,9.25) -- (12.75,9.75);
      \draw [short] (10.25,9.25) -- (10.25,9.75);
      \node [font=\large] at (7.75,9) {0};
      \node [font=\large] at (8.5,9) {1};
      \node [font=\large] at (9.5,9) {2};
      \node [font=\large] at (12.75,10) {\numprint{3600}};
      \node [font=\large] at (11,10) {900};
      \node [font=\large] at (8.5,10) {\numprint{2500}};
      \draw [short] (8.5,9.25) -- (8.5,9.75);
      \draw [short] (9.5,9.25) -- (9.5,9.75);
      \draw [short] (11,9.25) -- (11,9.75);
      \draw [short] (12,9.25) -- (12,9.75);
      \node [font=\large] at (10.25,9) {3};
      \node [font=\large] at (11,9) {4};
      \node [font=\large] at (12,9) {5};
      \node [font=\large] at (12.75,9) {6};
    \end{circuitikz}
  }%

\end{figure}
9\%\\
$VA=\frac{\numprint{2500}}{(1+0,09)^1}+\frac{900}{(1+0,09)^4}+\frac{\numprint{3600}}{(1+0,09)^6}=\numprint{5077,70}$\texteuro
\subsection{Annualità}
Sono valori che si ripetono a intervalli regolari di anno in anno\\
\begin{formula}
  $VF(A_n)=a*\frac{(a+r)^n-1}{r}$ \hspace{1cm}  $A_n$ = accumulazione finale\\
  \vspace{0.1cm}
  $a=VF(A_n)*\frac{r}{(1+r)^n-1}$ \hspace{1cm} a = ricerca dell'annualità media

\end{formula}
Es. $a=\numprint{10000}\texteuro*\frac{0,03}{(1+0,03)^4-1}=\numprint{2390,27}$\texteuro
\\
\vspace{0.2cm}
Rata mensile $=\frac{\numprint{2390,27}}{12+0,03*\frac{12+1}{2}}=192,25$\texteuro\\
\begin{tabular}{ll}
  $VA=\frac{VF}{(1+r)^n}=a*\frac{(1+r)^n-1}{r(1+r)^n}$ & VA = \numprint{800000}\texteuro                                                                        \\
                                                       & n = 20 anni                                                                                            \\
                                                       & r = 3\%                                                                                                \\
  $a=VA*\frac{r}{(1+r)^n-1}$                           & Rata = $\numprint{800000}\texteuro*\frac{(1+0,03)^{20}}{(1+0,03)^{20}-1}=\numprint{53772,56}\texteuro$ \\
\end{tabular}
\subsubsection{Annualità costanti posticipate limitate}
\subsection{Quota di ammortamento dei capitali - Rata mutuo}
\begin{formula}
  $Q_{am}=A_0*\frac{r(1+r)^n}{(1+r)^n-1}$
\end{formula}
\subsubsection{Piano di ammortamento}
Il debito può essere estinto in qualsiasi momento (debito residuo)
\\
Es. \numprint{100000}\texteuro, 7\%, 5 rate \\
$\numprint{24389,07}\texteuro*5=\numprint{121945,35}\texteuro$\\
\subsection{TAEN e TAN}
\subsubsection{TAEN}
Tasso Annuo Effettivo Globale, costo del prestito al netto di tasse e imposte
\subsubsection{TAN}
Tasso Annuo Nominale, non tiene conto delle spese
\subsection{Euribor}
Tasso Euribor: tasso interbancario di riferimento fornito da EMMI, ottenuto dalla media dei tassi di interesse applicati dalle banche europee. Incide sugli interessi a tasso variabile (composti da Euribor + spread)
\subsection{Annualità costanti posticipate illimitate - Perpetuity}
Annuity con flussi di cassa che continuano all'infinito, usato per esempio per la valutazione di un bene immobile.
\begin{formula}
  $VA(A_0)=\frac{a}{r}$\\
  $a=VA(A_0)*r$
\end{formula}
\section{Analisi degli investimenti 2}
\subsection{Investimento}
\begin{definition}
  Dare ai proprio risparmi una nuova veste investendoli in titoli o altri strumenti finanziari - Banca d'Italia
\end{definition}
\subsection{Progetti e beni di investimento}
Beni intermedi
\subsection{Input e Output}
\subsection{Effetti sugli investimenti}
\begin{itemize}
  \item Sui costi
  \item Sui ricavi
  \item Sul capitale circolante
  \item Congiunti (mix)
\end{itemize}
\subsection{Valutazione degli investimenti}
Es. progetto di sviluppo di un nuovo software
\subsection{Profili di analisi per le decisioni di investimento}
\begin{itemize}
  \item Profilo economico
  \item Profilo finanziario
\end{itemize}
\subsection{Criteri per la determinazione degli investimenti}
\begin{itemize}
  \item Dimensione dei flussi monetari\\$\Delta$ positivo tra flusso di ritorno e investimento
  \item Distribuzione temporale dei flussi\\\includegraphics[scale=0.4]{img1.png}
  \item Valore finanziario del tempo
\end{itemize}
\subsection{Metodologia del Valore Attuale Netto (VAN o NPV)}
Flussi di cassa a tempo $t_0$ con tasso di sconto adeguato
\begin{formula}
  NPV\footnote{Net Present Value}$=\sum_{t=0}^{n}\frac{C_t}{(1+r)^t}-I_0$
\end{formula}
\begin{important}
  Se NPV$<=0$ il progetto non è conveniente
\end{important}
\end{document}