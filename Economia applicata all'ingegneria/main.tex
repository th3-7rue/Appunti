\documentclass{report}

\usepackage[italian]{babel} % Imposta la lingua italiana
\usepackage{tcolorbox}
\usepackage{forest}
\usepackage{dirtree} % Aggiungi il pacchetto dirtree
\usepackage{graphicx} % Aggiungi il pacchetto graphicx
\usepackage{hyperref}
\usepackage{eurosym}
\usepackage{circuitikz}
\usepackage{amsmath}
\usepackage{emoji}
\hypersetup{
    colorlinks=true,
    linkcolor=black,
    filecolor=magenta,      
    urlcolor=cyan,
    pdftitle={Overleaf Example},
    pdfpagemode=FullScreen,
    }
% Definizione del nuovo ambiente 'important'
\newtcolorbox{important}[1][]{
  colback=yellow!10!white,
  colframe=red!50!black,
  title=Ricorda,
  #1
}
% Definizione del nuovo ambiente 'formula'
\newtcolorbox{formula}[1][]{
  colback=blue!5!white,
  colframe=blue!75!black,
  title=Formula,
  #1
}
% Definizione del nuovo ambiente 'definition'
\newtcolorbox{definition}[1][]{
  colback=green!5!white,
  colframe=green!75!black,
  title=Definizione,
  #1
}
\begin{document}

\begin{center}
  \vspace*{2cm}
  {\Huge Economia applicata all'ingegneria \par}
  \vspace{1cm}
  \includegraphics[width=0.5\textwidth]{logounibs.png}\par
  \vspace{1cm}
  {\Large Riccardo Rasori \par}
  \vspace{0.5cm}
  {\large A.A. 2024/2025 \par}
  \vspace{2cm}
\end{center}

\tableofcontents % Aggiungi l'indice

\chapter{Introduzione}
\section{Analisi degli investimenti}
\subsection{Introduzione alla matematica finanziara}
\textnormal{Valori nominali $\rightarrow$ anno corrente}
\newline
\textnormal{Valori reali $\rightarrow$ determinato anno (regolato a indice)}
\begin{formula}
  $Vk_t=(VC_t/IPC_t)*100$
\end{formula}
\vspace{1cm}
\dirtree{%
  .1 Tasso di interesse $\rightarrow$ prezzo della moneta $\rightarrow$ controllato da Banca Centrale .
  .2 Nominale .
  .2 Reale $\rightarrow$ $i_k=(ic-\gamma)/(1+\gamma)$  \newline{Nota: $\gamma$ rappresenta il tasso di inflazione.}.
}
\dirtree{%
  .1 La banca centrale controlla il prezzo della moneta .
  .2 Immettendo denaro tramite l'acquisto di titoli di stato $\rightarrow$ riduce il tasso .
  .2 Distruggendo denaro tramite la vendita di titoli di stato $\rightarrow$ aumenta il tasso .
}
\begin{formula}
  $C \rightarrow C(1+r)$\newline
  r $\rightarrow$ tasso di crescita \newline
  C $\rightarrow$ capitale
\end{formula}

\subsection{Costo opportunità del capitale}
\subsection{Interesse e montante semplice}
\begin{formula}
  $I=C*r*t$
\end{formula}
\begin{formula}
  $M=C(1+rt)$
  \newline
  M $\rightarrow$ montante $\rightarrow$ somma del capitale e degli interessi maturati nel tempo t
\end{formula}
Dimostrazione:\newline
$M=C+I=C+Crt=C(1+rt)$
\subsection{Montante semplice di rate stabili}
\begin{formula}
  $M=R(n+\frac{rn \pm 1}{2})$\newline
  +1 se la rata è anticipata, -1 se la rata è posticipata
\end{formula}
Es. 300€ canone mensile, 1,8\% saggio, 12 mesi\\
$M=300(12+0,018*\frac{12+1}{2})=3636,10$\euro
\subsection{Interesse e montante composto}
\begin{formula}
  $M_n=C(1+r)^n$  n = numero degli anni\\
  $C=\frac{R}{1+r}$
\end{formula}
$M_1=C(1+r)$\\
$M_2=M_1(1+r)$=$C(1+r)^2$\\
\vspace{0.1cm}

Es. Ho 1000\euro e li investo con +10\% ogni anno\\
\begin{figure}[!ht]
  \centering
  \resizebox{0.4\textwidth}{!}{%
    \begin{circuitikz}
      \tikzstyle{every node}=[font=\large]
      \draw [ color={rgb,255:red,40; green,32; blue,32}, short] (7.75,9.5) -- (12.75,9.5);
      \draw [short] (7.75,9.25) -- (7.75,9.75);
      \draw [short] (12.75,9.25) -- (12.75,9.75);
      \draw [short] (10.25,9.25) -- (10.25,9.75);
      \node [font=\large] at (7.75,9) {0};
      \node [font=\large] at (10.25,9) {1};
      \node [font=\large] at (12.75,9) {2};
      \node [font=\large] at (13,10) {1100+10\%(1100)};
      \node [font=\large] at (10.25,10) {1100};
      \node [font=\large] at (7.75,10) {1000};
    \end{circuitikz}
  }%
  \vspace{0.1cm}\\
  $C=\frac{M_n}{(1+r)^n}$\\
  \vspace{0.1cm}
  $q=1+r$\\
\end{figure}
\subsection{Valore futuro (VF)}
\begin{definition}
  È l'ammontare di una somma di denaro complessiva degli interessi in un determinato periodo.
\end{definition}
\subsection{Composizione degli interessi}
1000\euro, 2 anni, 10\%\\
$M_2=1000(1+0,1)^2=1210$\euro\\
\textbf{Interesse composto}
\subsection{Valore attuale (VA)}
Quanto devo investire oggi per avere 2000\euro \space tra un anno con saggio 11\%?\\
$VA=\frac{2000}{1+0,11}=1801,80$\euro
\begin{formula}
  $VA=\frac{FV}{(1+r)^n}$
\end{formula}
Desidero avere a disposizione 10000\euro\space per un viaggio negli States tra 4 anni dopo che mi laureo tutto pelato \emoji{neutral-face}\\
Quale somma dovrò accantonare ogni mese al saggio del 3\%?\\
Quale somma dovrò depositare sul conto corrente bancario oggi?\\
$\frac{10000\texteuro}{(1,03)^4}=8884,87$\texteuro
\end{document}