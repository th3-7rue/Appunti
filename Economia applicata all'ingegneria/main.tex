\documentclass{report}

\usepackage[italian]{babel} % Imposta la lingua italiana
\usepackage{tcolorbox}
\usepackage{forest}
\usepackage{dirtree} % Aggiungi il pacchetto dirtree
\usepackage{graphicx} % Aggiungi il pacchetto graphicx

% Definizione del nuovo ambiente 'important'
\newtcolorbox{important}[1][]{
  colback=yellow!10!white,
  colframe=red!50!black,
  title=Ricorda,
  #1
}
% Definizione del nuovo ambiente 'formula'
\newtcolorbox{formula}[1][]{
  colback=blue!5!white,
  colframe=blue!75!black,
  title=Formula,
  #1
}
\begin{document}

\begin{center}
  \vspace*{2cm}
  {\Huge Economia applicata all'ingegneria \par}
  \vspace{1cm}
  \includegraphics[width=0.5\textwidth]{logounibs.png}\par
  \vspace{1cm}
  {\Large Riccardo Rasori \par}
  \vspace{0.5cm}
  {\large A.A. 2024/2025 \par}
  \vspace{2cm}
\end{center}

\tableofcontents % Aggiungi l'indice

\chapter{Introduzione}
\section{19/02/2025}
\subsection{Analisi degli investimenti}
\subsubsection{Introduzione alla matematica finanziara}
\textnormal{Valori nominali $\rightarrow$ anno corrente}
\newline
\textnormal{Valori reali $\rightarrow$ determinato anno (regolato a indice)}
\begin{formula}
  $Vk_t=(VC_t/IPC_t)*100$
\end{formula}
\vspace{1cm}
\dirtree{%
  .1 Tasso di interesse $\rightarrow$ prezzo della moneta $\rightarrow$ controllato da Banca Centrale .
  .2 Nominale .
  .2 Reale $\rightarrow$ $i_k=(ic-\gamma)/(1+\gamma)$  \newline{Nota: $\gamma$ rappresenta il tasso di inflazione.}.
}
\dirtree{%
  .1 La banca centrale controlla il prezzo della moneta .
  .2 Immettendo denaro tramite l'acquisto di titoli di stato $\rightarrow$ riduce il tasso .
  .2 Distruggendo denaro tramite la vendita di titoli di stato $\rightarrow$ aumenta il tasso .
}
\begin{formula}
  $C \rightarrow C(1+r)$\newline
  r $\rightarrow$ tasso di crescita \newline
  C $\rightarrow$ capitale
\end{formula}

\subsubsection{Costo opportunità del capitale}
\subsubsection{Interesse e montante semplice}
\begin{formula}
  $I=C*r*t$
\end{formula}
\begin{formula}
  $M=C(1+rt)$
  \newline
  M $\rightarrow$ montante $\rightarrow$ somma del capitale e degli interessi maturati nel tempo t
\end{formula}
Dimostrazione:\newline
$M=C+I=C+Crt=C(1+rt)$
\subsubsection{Montante semplice di rate stabili}
\begin{formula}
  $M=R(n+\frac{rn \pm 1}{2})$\newline
  +1 se la rata è anticipata, -1 se la rata è posticipata
\end{formula}
\subsubsection{Interesse e montante composto}
\end{document}