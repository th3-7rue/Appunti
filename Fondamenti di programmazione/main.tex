\documentclass{article}

\usepackage{tcolorbox}
\usepackage{forest}
\usepackage{dirtree} % Aggiungi il pacchetto dirtree

% Definizione del nuovo ambiente 'important'
\newtcolorbox{important}[1][]{
  colback=yellow!10!white,
  colframe=red!50!black,
  title=Ricorda,
  #1
}

\begin{document}

\title{Fondamenti di Programmazione}
\author{Riccardo Rasori}
\date{\today}

\maketitle

\section{17/02/2025}

\dirtree{%
    .1 Classe .
    .2 Metodi .
    .2 Strutture .
}

\begin{important}
    Per compilare un file .java, usare il comando \texttt{javac nomeFile.java}.
    Tale comanda genera il file \texttt{nomeFile.class}.
    Per eseguirlo, usare il comando \texttt{java nomeFile}.
\end{important}




\end{document}