\documentclass{report}

\usepackage[italian]{babel} % Imposta la lingua italiana
\usepackage{tcolorbox}
\usepackage{forest}
\usepackage{dirtree} % Aggiungi il pacchetto dirtree
\usepackage{graphicx} % Aggiungi il pacchetto graphicx

% Definizione del nuovo ambiente 'important'
\newtcolorbox{important}[1][]{
  colback=yellow!10!white,
  colframe=red!50!black,
  title=Ricorda,
  #1
}

\begin{document}

\begin{center}
  \vspace*{2cm}
  {\Huge Fondamenti di Programmazione \par}
  \vspace{1cm}
  \includegraphics[width=0.5\textwidth]{logounibs.png}\par
  \vspace{1cm}
  {\Large Riccardo Rasori \par}
  \vspace{0.5cm}
  {\large A.A. 2024/2025 \par}
  \vspace{2cm}
\end{center}

\tableofcontents % Aggiungi l'indice

\chapter{Introduzione}
\section{17/02/2025}

\dirtree{%
  .1 Classe .
  .2 Metodi .
  .2 Strutture .
}

\begin{important}
  Per compilare un file .java, usare il comando \texttt{javac nomeFile.java}.
  Tale comanda genera il file \texttt{nomeFile.class}.
  Per eseguirlo, usare il comando \texttt{java nomeFile}.
\end{important}

\section{18/02/2025}
\dirtree{%
  .1 Tipi di errore .
  .2 Compile time .
  .3 Syntax error .
  .2 Run time .
  .3 Logic error .
  .3 Errore di esecuzione .
}
\vspace{0.5cm}
\textsc{Javadoc}\newline
\texttt{/** \newline
  * Questo è un commento Javadoc \newline
  * @param x è un parametro \newline
  */\newline}
\vspace{0.2cm}
\textnormal{Per generare la documentazione, usare il comando \texttt{javadoc nomeFile.java}.}
\end{document}